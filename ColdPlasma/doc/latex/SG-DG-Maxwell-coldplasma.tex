
\documentclass[final,leqno]{siamltex704}
%\documentclass[leqno]{siamltex704}
\usepackage{epsfig}
\usepackage{amsmath,bm}
\usepackage{amssymb}
\usepackage{float}
\usepackage{tikz}
\usepackage{graphicx}
\usepackage[notcite,notref]{showkeys}
\newtheorem{algorithm}{Weak Galerkin Algorithm}
\newcommand{\bq}{{\bf q}}
\newcommand{\bn}{{\bf n}}
\newcommand{\bx}{{\bf x}}
\newcommand{\bv}{{\bf v}}
\def\bbb{{\bf b}}
\def\T{{\mathcal T}}
\def\E{{\mathcal E}}
\def\V{{\mathcal V}}
\def\W{{\mathcal W}}
\def\l{{\langle}}
\def\r{{\rangle}}
\def\jump#1{{[\![#1[\!]}}
\def\bbF{{\bf F}}
\def\bbf{{\bf f}}
\def\bn{{\bf n}}
\def\bq{{\bf q}}
\def\bV{{\bf V}}
\def\bu{{\bf u}}
\def\bv{{\bf v}}
\def\bw{{\bf w}}
\def\br{{\bf r}}
\def\bs{{\bf s}}
\def\bbQ{\mathbb{Q}}
\def\bfQ{\bf{Q}}
\def\bcurl{ \textbf{curl }}
\def\bE{{\bf E}}
\def\bB{{\bf B}}
\def\bx{{\bf x}}
\def\bU{{\bf U}}
\def\bV{{\bf V}}
\def\bJ{{\bf J}}
\def\be{{\bf e}}

\def\ljump{{[\![}}
\def\rjump{{]\!]}}

\def\lavg{{\{\!\{}}
\def\ravg{{\}\!\}}}

\def\aa{\mathfrak{a}}
\def\bbQ{\mathbb{Q}}
\newcommand{\pT}{{\partial T}}
\newtheorem{defi}{Definition}[section]
\def\3bar{{|\hspace{-.02in}|\hspace{-.02in}|}}
%\setlength{\textwidth}{6truein} \setlength{\textheight}{8truein}
%\voffset=-0.55truein
%\hoffset=-0.65truein
\renewcommand{\ldots}{\dotsc}
\setlength{\parskip}{1\parskip}

\title{Discontinuous Galerkin Sparse Grids Methods for Maxwell's Equations of A Cold Plasma}


\author{
}

\begin{document}

\maketitle

\begin{abstract}
\end{abstract}

\begin{keywords}
Maxwell's equations, time domain, discontinuous Galerkin method, sparse grids methods.
\end{keywords}

\begin{AMS}
Primary: 65N15, 65N30; Secondary: 35J50
\end{AMS}
\pagestyle{myheadings}

\section{Introduction}
Time-harmonic Maxwell's equations with dielectric tensor $\epsilon^{\nu}\in C^0(\Omega : \mathcal{M}^{d\times d})$:
\begin{eqnarray}
\nabla\times\nabla\times{\bE}-\frac{\omega^2}{c^2}\epsilon^{\nu}\bE = 0
\end{eqnarray}
with
\begin{eqnarray*}
\epsilon^{\nu}=\begin{pmatrix}
1-\dfrac{\tilde{\omega}\omega_p^2}{\omega(\tilde{\omega}^2-\omega_c^2)}  & i\dfrac{\omega_c\omega_p^2}{\omega(\tilde{\omega}^2-\omega_c^2)} & 0\\
-i\dfrac{\omega_c\omega_p^2}{\omega(\tilde{\omega}^2-\omega_c^2)}& 1-\dfrac{\tilde{\omega}\omega_p^2}{\omega(\tilde{\omega}^2-\omega_c^2)} & 0\\
0 & 0 & 1-\dfrac{\omega_p^2}{\omega\tilde{\omega}}
\end{pmatrix}.
\end{eqnarray*}
Here, $\omega_p({\bf x})^2=\frac{e^2N_e({\bf x})}{c^2\epsilon_0}$, $\omega_c({\bf x})=\frac{e|{\bf B}_0({\bf x})|}{m_e}$ and $\tilde{\omega}=\omega+i\nu.$ The collision frequency is $\nu>0$. it corresponds to friction on a bath of static ions. It is an extremely small quantity because the plasma has very low collisionality $\nu\approx 10^{-7}$ in a fusion plasma.

In this equation, the dispersive media whose physical parameters are wavelength dependent.
\begin{eqnarray}
\nabla\times\nabla\times\bu-\kappa^2\bu =\bbf.
\end{eqnarray}
Formulation:
\begin{eqnarray}
A(\bu_h,\bv)&=&(\nabla_h\times \bu_h,\nabla_h\times \bv)-k^2(\bu_h,\bv)\notag\\
&&-\langle\lavg\nabla_h\times \bu_h\ravg,\ljump \bv\rjump\rangle-\langle\ljump \bu_h\rjump,\lavg\nabla_h\times \bv\ravg\rangle
+\alpha h^{-1}\langle\ljump \bu_h\rjump,\ljump \bv\rjump\rangle
\end{eqnarray}
and
\begin{eqnarray}
A(\bu_h,\bv)=(\bbf,\bv).
\end{eqnarray}

%Let the test and trial functions as: 
%\begin{eqnarray*}
%\bm{\phi_{j,m}^l}:=\bigg(\phi^{l_1}_{j_1,m_1}(x_1)\phi^{l_2}_{j_2,m_2}(x_2)\cdots\bigg)\mbox{ and }
%\bm{\phi_{i,n}^\ell}:=\bigg(\phi^{\ell_1}_{i_1,n_1}(x_1)\phi^{\ell_2}_{i_2,n_2}(x_2)\cdots\bigg)
%\end{eqnarray*}
%where $|\bm{l}|_1\le n$ and $|\bm{\ell}|_1\le n.$
%Then the operator $\mathcal{A}$ acts on them are as follows:
%\begin{eqnarray*}
%\mathcal{A}(\bm{\phi_{j,m}^l},\bm{\phi_{i,n}^\ell})=A_1(\phi^{l_1}_{j_1,m_1},\phi^{\ell_1}_{i_1,n_1})\bigotimes A_2(\phi^{l_2}_{j_2,m_2},\phi^{\ell_2}_{i_2,n_2})\bigotimes\cdots
%\end{eqnarray*}
%Through mapping $(\bm{l,j,m})\to J$ and $(\bm{\ell,i,n})\to I$, we generate the entry $(\mathbb{A})_{I,J}$.
%
%\begin{eqnarray*}
%\hat{\bE} &=& \frac{\bE^++\bE^-}{2},\ \hat{\bB} = \frac{\bB^++\bB^-}{2}\\
%\hat{\bE} &=& \bE^+,\ \hat{\bB} = \bB^-,\mbox{ or }\hat{\bE} = \bE^-,\ \hat{\bB} = \bB^+\\
%\hat{\bE} &=& \frac{\bE^++\bE^-}{2}+(\bB^+\times\bn^++\bB^-\times\bn^-),\ \hat{\bB} = \frac{\bB^++\bB^-}{2}-(\bE^+\times\bn^++\bE^-\times\bn^-)
%\end{eqnarray*}
%-------------------------------------------
% Numerical Examples
%-------------------------------------------
\section{Numerical Examples}
\subsection{Test 1}
Let $\Omega=[0,1]^3$ and the exact solution is given by
\begin{eqnarray}
\bu(x,y,z)=\begin{pmatrix}
\sin(\pi y)\sin(\pi z)\\
\sin(\pi z)\sin(\pi x)\\
\sin(\pi x)\sin(\pi y)
\end{pmatrix}.
\end{eqnarray}
\begin{table}[htbp]
  \centering
  \caption{High-order Maxwell2}
    \begin{tabular}{rrrrr}
1/h & Max Error & Rate & L2 Error & Rate\\
\multicolumn{5} {c}{k=1}	\\
    \end{tabular}%
\end{table}%


\begin{thebibliography}{99}
\bibitem{Dolean2010}
V. Dolean, H. Fahs, L. Fezoui, and S. Lanteri. Locally implicit discontinous Galerkin method for time domain electromagnetics, J. Comput. Phys., 229 (2010): 512-526.

\bibitem{GuoCheng2016}
W. Guo and Y. Cheng. A Sparse Grid Discontinuous Galerkin Method for High-Dimensional Transport Equations and Its Application to Kinetic Simulations. SIAM J. Sci. Comput., 38(6), A3381-A3409. 

\bibitem{Kopriva2002}
D.A.Kopriva, S.L.Woodruff and M.Y.Hussaini. Computation of electromagnetic scattering with a non-conforming discontinuous spectral element method, Int. J. Numr. Meth. Engng, 53 (2002): 105-122.

\bibitem{LiHesthaven2014}
J. Li and J.S.Hesthaven. Analysis and application of the nodal discontinuous Galerkin method for wave propagation in metamaterials, J. Comput. Phys., 258 (2014): 915-930.

\bibitem{ShuOsher1988}
C.W. Shu and S. Osher. Efficient implementation of essentially non-oscillatory shock-capturing schemes. J. Comput.
Phys., 77 (1988):439-471.


\bibitem{Warburton2000}
T. Warburton. Application of the discontinuous Galerkin method to Maxwell's equations using unstructured polymorphic hp-dinite elements, In Discontinuous Galerkin Methods, 2000, 451-458.

\bibitem{XieWangZhang2013}
Z. Xie, B. Wang and Z. Zhang. Space-Time discontinuous Galerkin method for Maxwell's equations, Commun. Comput.Phys., 14 (2013): 916-939.



\end{thebibliography}

\end{document}


