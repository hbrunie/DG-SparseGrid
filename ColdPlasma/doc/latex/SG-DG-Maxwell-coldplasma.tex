
\documentclass[final,leqno]{siamltex704}
%\documentclass[leqno]{siamltex704}
\usepackage{epsfig}
\usepackage{amsmath,bm}
\usepackage{amssymb}
\usepackage{float}
\usepackage{tikz}
\usepackage{graphicx}
\usepackage[notcite,notref]{showkeys}
\newtheorem{algorithm}{Weak Galerkin Algorithm}
\newcommand{\bq}{{\bf q}}
\newcommand{\bn}{{\bf n}}
\newcommand{\bx}{{\bf x}}
\newcommand{\bv}{{\bf v}}
\def\bbb{{\bf b}}
\def\T{{\mathcal T}}
\def\E{{\mathcal E}}
\def\V{{\mathcal V}}
\def\W{{\mathcal W}}
\def\l{{\langle}}
\def\r{{\rangle}}
\def\jump#1{{[\![#1[\!]}}
\def\bbF{{\bf F}}
\def\bbf{{\bf f}}
\def\bn{{\bf n}}
\def\bq{{\bf q}}
\def\bV{{\bf V}}
\def\bu{{\bf u}}
\def\bv{{\bf v}}
\def\bw{{\bf w}}
\def\br{{\bf r}}
\def\bs{{\bf s}}
\def\bbQ{\mathbb{Q}}
\def\bfQ{\bf{Q}}
\def\bcurl{ \textbf{curl }}
\def\bE{{\bf E}}
\def\bB{{\bf B}}
\def\bx{{\bf x}}
\def\bU{{\bf U}}
\def\bV{{\bf V}}
\def\bJ{{\bf J}}
\def\be{{\bf e}}

\def\ljump{{[\![}}
\def\rjump{{]\!]}}

\def\lavg{{\{\!\{}}
\def\ravg{{\}\!\}}}

\def\aa{\mathfrak{a}}
\def\bbQ{\mathbb{Q}}
\newcommand{\pT}{{\partial T}}
\newtheorem{defi}{Definition}[section]
\def\3bar{{|\hspace{-.02in}|\hspace{-.02in}|}}
%\setlength{\textwidth}{6truein} \setlength{\textheight}{8truein}
%\voffset=-0.55truein
%\hoffset=-0.65truein
\renewcommand{\ldots}{\dotsc}
\setlength{\parskip}{1\parskip}

\title{Discontinuous Galerkin Sparse Grids Methods for Maxwell's Equations of A Cold Plasma}


\author{
}

\begin{document}

\maketitle

\begin{abstract}
\end{abstract}

\begin{keywords}
Maxwell's equations, time domain, discontinuous Galerkin method, sparse grids methods.
\end{keywords}

\begin{AMS}
Primary: 65N15, 65N30; Secondary: 35J50
\end{AMS}
\pagestyle{myheadings}

\section{Introduction}
Time-harmonic Maxwell's equations with dielectric tensor $\epsilon^{\nu}\in C^0(\Omega : \mathcal{M}^{d\times d})$:
\begin{eqnarray}
\nabla\times\nabla\times{\bE}-\frac{\omega^2}{c^2}\epsilon^{\nu}\bE = 0
\end{eqnarray}
with
\begin{eqnarray*}
\epsilon^{\nu}=\begin{pmatrix}
1-\dfrac{\tilde{\omega}\omega_p^2}{\omega(\tilde{\omega}^2-\omega_c^2)}  & i\dfrac{\omega_c\omega_p^2}{\omega(\tilde{\omega}^2-\omega_c^2)} & 0\\
-i\dfrac{\omega_c\omega_p^2}{\omega(\tilde{\omega}^2-\omega_c^2)}& 1-\dfrac{\tilde{\omega}\omega_p^2}{\omega(\tilde{\omega}^2-\omega_c^2)} & 0\\
0 & 0 & 1-\dfrac{\omega_p^2}{\omega\tilde{\omega}}
\end{pmatrix}.
\end{eqnarray*}
Here, $\omega_p({\bf x})^2=\frac{e^2N_e({\bf x})}{c^2\epsilon_0}$, $\omega_c({\bf x})=\frac{e|{\bf B}_0({\bf x})|}{m_e}$ and $\tilde{\omega}=\omega+i\nu.$ The collision frequency is $\nu>0$. it corresponds to friction on a bath of static ions. It is an extremely small quantity because the plasma has very low collisionality $\nu\approx 10^{-7}$ in a fusion plasma.

In this equation, the dispersive media whose physical parameters are wavelength dependent.


%------Equation--------------
Find the vector unknown $\bu$ that satisfies
\begin{eqnarray}
\nabla\times\nabla\times\bu-\kappa^2\bu &=&\bbf,\mbox{ in }\Omega,\label{eq:pde1}\\
\bn\times\bu &=& 0, \mbox{ on }\Gamma.\label{eq:pde1-bc}
\end{eqnarray}
Here, $\Omega$ is taken to be an open bounded Lipschitz polyhedron in $\mathbb{R}^3$ with boundary $\Gamma=\partial\Omega$ and outward normal unite vector $\bn.$ For simplicity, we assume $\Omega$ to be simply-connected and $\Gamma$ to be connected. The right-hand side $\bbf$ is a given external source field in $L^2(\Omega)^3.$ By introducing the Sobolev space
\begin{eqnarray*}
H_0(\text{curl};\Omega):=\{\bv\in L^2(\Omega)^3:\nabla\times\bv\in L^2(\Omega)^3,\bn\times\bv=0,\mbox{ on }\Gamma\},
\end{eqnarray*}
the weak form is given by: find $\bu\in H_0(\text{curl};\Omega)$ such that
\begin{eqnarray*}
A(\bu,\bv):=\int_{\Omega}\bigg((\nabla\times\bu)\cdot(\nabla\times\bv)+\bu\cdot\bv\bigg)dx=\int_{\Omega}\bbf\cdot\bv dx,\ \forall\bv\in H_0(\text{curl};\Omega).
\end{eqnarray*}

%-------Advantage of DG---------------
The main motivation for using a discontinuous Galerkin (DG) approach for the numerical approximation of (\ref{eq:pde1})-(\ref{eq:pde1-bc}) is that DG methods, being based on discontinuous finite element spaces, can easily handle non-conforming meshes which contain hanging nodes and, in principle, local spaces of different polynomial orders. Moreover, the implementation of discontinuous elements can be based on standard shape functions, without the need to employ curl-conforming elemental mappings - a convenience that is particularly advantageous for high-order elements and that is not straightforwardly shared by standard edge or face elements commonly used in computational electromagnetics. A further benefit of DG methods is that inhomogeneous Dirichlet boundary conditions can easily be incorporated within the scheme, without the need to explicitly evaluate edge- and face-element interpolation operators.


%----Sparse grids


%----organization of this paper


\section{Discontinous Galerkin discretization}
In this section, we consider the interior penalty DG discretization of (\ref{eq:pde1})-(\ref{eq:pde1-bc}). We first introduce the following notations.

\vspace{5pt}
% Triangulation
Let $\mathcal{T}_h$ be a conforming, shape-regular partition of domain $\Omega$. Denote $h=\max_{T\in\mathcal{T}_h}h_T$, with $h_T$ defined as the diameter of the element $T\in\mathcal{T}_h$. Let $\mathcal{F}_h,\ \mathcal{F}_h^I,\ \mathcal{F}_h^B$ denote the set of all faces, the interior faces and the boundary faces respectively. 
For a piecewise smooth vector-valued function $\bv$, we shall introduce the following trace operators. Let $e\in\mathcal{F}_h^I$ be an interior face shared by two neighboring elements $K^+$ and $K^-$; $\bn^\pm$ be the unit outward normal vectors on the boundaries $\partial K^\pm$, and we can define average and jump across a face $e\in\mathcal{F}_h^I$ as following
\begin{eqnarray*}
\ljump \bu \rjump_{\tau}={\bf n}^+\times\bu^++{\bf n}^-\times\bu^-,\mbox{ and }\lavg \bu\ravg=\frac{\bu^++\bu^-}{2}.
\end{eqnarray*}
If $e\in\mathcal{F}_h^B$, the average and jump are defined as
\begin{eqnarray*}
\ljump \bu \rjump_{\tau}={\bf n}\times\bu,\mbox{ and } \lavg\bu\ravg=\bu.
\end{eqnarray*}


For a given partition $\mathcal{T}_h$ and an approximation degree $k\ge 1$, we introduce the following finite element space
\begin{eqnarray}
V_h:=\{\bv\in L^2(\Omega)^3:\bv|_T\in P^k(T)^3,T\in\mathcal{T}_h\},
\end{eqnarray}
where $P^k(T)$ denotes the space of polynomials of total degree at most $k$ on $T$. With this notation, we consider the following interior penalty DG method: find $\bu_h\in V_h$ such that
%---------------------------
% Numerical Scheme
%---------------------------
\begin{eqnarray}
A_h(\bu_h,\bv)=(\bbf,\bv),
\end{eqnarray}
with $A_h(\cdot,\cdot)$ is given by
\begin{eqnarray}
A_h(\bu_h,\bv)&=&(\nabla_h\times \bu_h,\nabla_h\times \bv)-k^2(\bu_h,\bv)\notag\\
&-&\langle\lavg\nabla_h\times \bu_h\ravg,\ljump \bv\rjump_{\tau}\rangle_{\mathcal{F}_h}-\langle\ljump \bu_h\rjump_{\tau},\lavg\nabla_h\times \bv\ravg\rangle_{\mathcal{F}_h}
+\alpha h^{-1}\langle\ljump \bu_h\rjump_{\tau},\ljump \bv\rjump_{\tau}\rangle_{\mathcal{F}_h}.
\end{eqnarray}
Here, we use $\nabla_h$ to denote the elementwise application of the operator $\nabla$, and $\langle \bv, \bw\rangle_{\mathcal{F}_h}:=\sum_{e\in\mathcal{F}_h}\int_e \bv\cdot\bw ds.$
Here $\alpha\ge \alpha_{\min}$ is a positive constant that is independent of the mesh size. The threshold value $\alpha_{\min}$ only depends on the shape-regularity of the mesh and the approximation order $k$.

%---------------------
% Norm
%---------------------
Here we define the following DG seminorm and norm on $V(h)$, respectively:
\begin{eqnarray}
%|\bv|_{\text{DG}}^2:=\|\nabla_h\times\bv\|_0^2+\|h^{-1/2}\ljump\bv\rjump_{\tau}\|_{0,\mathcal{F}_h}^2,\ 
\3bar\bv\3bar^2:=\|\bv\|_0^2+\|\nabla_h\times\bv\|_0^2+\|h^{-1/2}\ljump\bv\rjump_{\tau}\|_{0,\mathcal{F}_h}^2.
\end{eqnarray}

\begin{eqnarray*}
A_h(\bv,\bv)=\|\nabla_h\times\bv\|_0^2-k^2\|\bv\|_0^2-2\langle\lavg\nabla_h\times\bv\ravg,\ljump\bv\rjump_{\tau}\rangle_{\mathcal{F}_h}+\alpha h^{-1}\|\ljump\bv\rjump_{\tau}\|_{\mathcal{F}_h}^2
\end{eqnarray*}
Here
\begin{eqnarray*}
\langle\lavg\nabla_h\times\bv\ravg,\ljump\bv\rjump_{\tau}\rangle_{\mathcal{F}_h}&\le& 
\frac{\epsilon h\|\lavg\nabla_h\times\bv\ravg\|_{\mathcal{F}_h}^2}{2}
+\frac{h^{-1}\|\ljump\bv\rjump_{\tau}\|_{\mathcal{F}_h}^2}{2\epsilon}\\
&\le& \frac{\epsilon(C_{\text{inv}}+1)}{2}\|\nabla_h\times\bv\|^2+\frac{h^{-1}\|\ljump\bv\rjump_{\tau}\|_{\mathcal{F}_h}^2}{2\epsilon}
\end{eqnarray*}
Thus
\begin{eqnarray*}
A_h(\bv,\bv)\ge \big(1-\epsilon(C_{\text{inv}}-1)\big)\|\nabla_h\times\bv_{\tau}\|_0^2+(\alpha-\frac{1}{2\epsilon})h^{-1}\|\ljump\bv\rjump\|_{\mathcal{F}_h}^2-k^2\|\bv\|_0^2
\end{eqnarray*}
%---------------------
% theorem
%---------------------
\begin{theorem}
Assume that $\bu\in H^s(\Omega)^3,\ \nabla\times\bu\in H^s(\Omega)^3$, for $s>1/2$. Furthermore, let $\bu_h$ denote the DG approximation with $\alpha\ge \alpha_{\min}.$ Then there is a mesh size $h_0>0$ such that for $0<h\le h_0$, we have the optimal a priori error bound
\begin{eqnarray}
\|\bu-\bu_h\|_{\text{DG}}\le Ch^{\min\{s,k\}}\bigg(\|\bu\|_s+\|\nabla\times\bu\|_s\bigg),
\end{eqnarray}
with a constant $C>0$ independent of the mesh size.
\end{theorem}

\begin{theorem}
Assume that the exact solution $\bu$ satisfies the regularity assumption
\begin{eqnarray}
\bu\in H^{s+1}(\Omega)^3,\ \nabla\times\bu\in H^s(\Omega)^3,
\end{eqnarray}
for $s>1/2$. Let $\bu_h$ denote the DG approximation obtained by (?) with $\alpha\ge\alpha_{\min}$. Then we have the following error bound
\begin{eqnarray}
\|\bu-\bu_h\|\le Ch^{\min{s,k}+1}\bigg(\|\bu\|_{s+1}+\|\nabla\times\bu\|_s\bigg),
\end{eqnarray}
with a constant $C>0$ independent of the mesh size.
\end{theorem}

%Let the test and trial functions as: 
%\begin{eqnarray*}
%\bm{\phi_{i,n}^\ell}:=\bigg(\phi^{\ell_1}_{i_1,n_1}(x_1)\phi^{\ell_2}_{i_2,n_2}(x_2)\cdots\bigg)\mbox{ and }
%\bm{\phi_{j,m}^l}:=\bigg(\phi^{l_1}_{j_1,m_1}(x_1)\phi^{l_2}_{j_2,m_2}(x_2)\cdots\bigg)
%\end{eqnarray*}
%where $|\bm{l}|_1\le n$ and $|\bm{\ell}|_1\le n.$
%Then the operator $\mathcal{A}$ acts on them are as follows:
%\begin{eqnarray*}
%\mathcal{A}(\bm{\phi_{i,n}^l},\bm{\phi_{j,m}^\ell})=A_1(\phi^{l_1}_{i_1,n_1},\phi^{\ell_1}_{j_1,m_1})\bigotimes A_2(\phi^{l_2}_{i_2,n_2},\phi^{\ell_2}_{j_2,m_2})\bigotimes\cdots
%\end{eqnarray*}
%Through mapping $(\bm{\ell,i,n})\to I$ and $(\bm{l,j,m})\to J$, we generate the entry $(\mathbb{A})_{I,J}$.
%
%\begin{eqnarray*}
%\hat{\bE} &=& \frac{\bE^++\bE^-}{2},\ \hat{\bB} = \frac{\bB^++\bB^-}{2}\\
%\hat{\bE} &=& \bE^+,\ \hat{\bB} = \bB^-,\mbox{ or }\hat{\bE} = \bE^-,\ \hat{\bB} = \bB^+\\
%\hat{\bE} &=& \frac{\bE^++\bE^-}{2}+(\bB^+\times\bn^++\bB^-\times\bn^-),\ \hat{\bB} = \frac{\bB^++\bB^-}{2}-(\bE^+\times\bn^++\bE^-\times\bn^-)
%\end{eqnarray*}


\section{Multiwavelet DG Method}
In this section, we summarize some properties of the multiwavelet bases derived and introduce notations. 

For $k=1,2,\dots,$ and $n=0,1,2,\dots,$ we define $V_n^k$ as a space of piecewise polynomial functions,
\begin{eqnarray}
V_n^k=\{f:f\in\Pi_k(I_{n,l},\text{ for }l=0,\dots,2^n-1,\text{ and }supp(f)=I_l^n)\},
\end{eqnarray}
where $\Pi_k(I_l^n)$ is the space of all polynomial of degree less than $k$ on the interval $I_l^n=[2^{-n}l,2^{-n}(l+1)]$. The space $V_n^k$ has dimension $2^nk$ and has the following nested property,
\begin{eqnarray}
V_0^k\subset V_1^k\subset\cdots\subset V_n^k\subset \cdots.
\end{eqnarray}
The multiwavelet subspace $W_n^k$, $n=0,1,2,\dots,$ is defined as the orthogonal complement of $V_n^k$ in $V_{n+1}^k$ or
\begin{eqnarray}
V_n^k\oplus W_n^k=V_{n+1}^k,\ W_n^k\perp V_n^k,
\end{eqnarray}
where the norm is defined as $\|f\|=\int_0^1 f(x)g(x)dx.$ 

Given a basis $\phi_0,\dots,\phi_{k-1}$ of $V_0^k$, the space $V_n^k$ is spanned by $2^nk$ functions which are obtained from $\phi_0,\dots,\phi_{k-1}$ by dilation and translation,
\begin{eqnarray}
\phi_{jl}^n(x)=2^{n/2}\phi_j(2^nx-l),\ j=0,\dots,k-1,\ l=0,\dots,2^n-1.
\end{eqnarray}
It the piecewise polynomial functions, $\psi_0,\cdots,\psi_{k-1}$ form an orthonormal basis for $W_0^k$, then by dilation and translation the space $W_n^k$ is spanned by $2^nk$ functions
\begin{eqnarray}
\psi_{jl}^n=2^{n/2}\psi_j(2^nx-l),\ j=0,\dots,k-1,\ l=0,\dots,2^n-1.
\end{eqnarray}

%%---------------------
%% FEM space
%%---------------------
%Scaling functions
%\begin{eqnarray*}
%\phi_{\ell,j}^n = 2^{n/2}\sqrt{2j-1}P_j(2^{n+1}-2\ell-1).
%\end{eqnarray*}
%
%
%
%
%
%Here we have
%\begin{eqnarray*}
%\nabla_h\times\bv&=&(\partial_y v_3-\partial_zv_2,\partial_zv_1-\partial_xv_3,\partial_xv_2-\partial_yv_1)
%\end{eqnarray*}
%
%\begin{eqnarray*}
%(\nabla_h\times\bv,\nabla_h\times\bw)&=&\sum_{T\in\mathcal{T}_h}\int_T
%	\begin{pmatrix}
%	\partial_y v_3-\partial_zv_2\\ \partial_zv_1-\partial_xv_3\\ \partial_xv_2-\partial_yv_1
%	\end{pmatrix}\cdot
%	\begin{pmatrix}
%	\partial_y w_3-\partial_zw_2\\ \partial_zw_1-\partial_xw_3\\ \partial_xw_2-\partial_yw_1
%	\end{pmatrix}dT \\&=&\sum_{T\in\mathcal{T}_h}\int_T
%\begin{pmatrix}
%\partial_y w_1\partial_y v_1+\partial_z w_1\partial_z v_1 & -\partial_y w_1\partial_x v_2 & -\partial_zw_1\partial_xv_3\\
%-\partial_xw_2\partial_yv_1 & \partial_xw_2\partial_xv_2+\partial_zw_2\partial_zv_2 &-\partial_zw_2\partial_yv_3\\
%-\partial_xw_3\partial_zv_1 & -\partial_yw_3\partial_zv_2 & \partial_xw_3\partial_xv_3+\partial_yw_3\partial_yv_3
%\end{pmatrix}dT\\
%&:=&\mathbb{K}.
%\end{eqnarray*}
%We shall compute each entry as follows:
%\begin{eqnarray*}
%\mathbb{K}_{11} &=& \int_T\partial_y w_1\partial_y v_1+\partial_z w_1\partial_z v_1dT\\
%&=&\int_{T_x}w_1(x)v_1(x)dx\int_{T_y} w_1'(y)v_1'(y)dy\int_{T_z}w_1(z)v_1(z)dz\\
%&&+\int_{T_x}w_1(x)v_1(x)dx\int_{T_y} w_1(y)v_1(y)dy\int_{T_z}w'_1(z)v'_1(z)dz\\
%&=&\mathbb{I}_x\bigotimes\mathbb{GG}_y\bigotimes\mathbb{I}_z+\mathbb{I}_x\bigotimes\mathbb{I}_y\bigotimes\mathbb{GG}_z
%\end{eqnarray*}
%with $\mathbb{GG}:=\int w'v'dx.$
%Also, we have
%\begin{eqnarray*}
%\mathbb{K}_{12}&=& \int_T -\partial_y w_1\partial_x v_2dT\\
%&=&\int_{T_x}w_1(x)v_2'(x)dx\int_{T_y}w_1'(y)v_2(y)dy\int_{T_z}w_1(z)v_2(z)dz\\
%&=&\mathbb{G}\bigotimes\mathbb{G}^\top\bigotimes\mathbb{I}
%\end{eqnarray*}
%with $\mathbb{G}=\int uv'dx.$
%\begin{eqnarray*}
%\langle\lavg\nabla\times\bu_h\ravg,\ljump\bv\rjump \rangle &=& \int_{\partial T}
%	\begin{pmatrix}
%	\partial_y u_3-\partial_zu_2\\ \partial_zu_1-\partial_xu_3\\ \partial_xu_2-\partial_yu_1
%	\end{pmatrix}\cdot
%	\begin{pmatrix}
%	n_2 v_3-n_z v_2\\ n_z v_1-n_x v_3\\ n_x v_2-n_yv_1
%	\end{pmatrix}ds\\
%	&=&\int_{\partial T}
%	\begin{pmatrix}
%	n_y v_1\partial_y u_1+n_z v_1\partial_z u_1 & -n_y v_1\partial_x u_2 & -n_zv_1\partial_xu_3\\
%	-n_xv_2\partial_yu_1 & n_xv_2\partial_xu_2+n_zv_2\partial_zu_2 &-n_zv_2\partial_yu_3\\
%	-n_xv_3\partial_zu_1 & -n_yv_3\partial_zu_2 & n_xv_3\partial_xu_3+n_yv_3\partial_yu_3
%	\end{pmatrix}ds
%\end{eqnarray*}
%So on $\partial T_x$ with $\bn=(1,0,0)$ or $\bn=(-1,0,0)$, we have
%\begin{eqnarray*}
%\int_{\partial T_x}
%	\begin{pmatrix}
%	0 & 0 & 0\\
%	-n_xv_2\partial_yu_1 & n_xv_2\partial_xu_2 &0\\
%	-n_xv_3\partial_zu_1 & 0 & n_xv_3\partial_xu_3
%	\end{pmatrix}ds
%	&=&
%\int_{\partial T_x}
%	\begin{pmatrix}
%	0 & 0 & 0\\
%	-\ljump v_2\rjump_x\lavg\partial_yu_1\ravg & \ljump v_2\rjump_x\lavg\partial_xu_2\ravg &0\\
%	-\ljump v_3\rjump_x\lavg\partial_zu_1\ravg & 0 & \ljump v_3\rjump_x\lavg\partial_xu_3\ravg
%	\end{pmatrix}	\\
%	&=&
%	\begin{pmatrix}
%	0 & 0 & 0\\
%	-\mathbb{K}\bigotimes\mathbb{G}\bigotimes\mathbb{I}&\mathbb{L}\bigotimes\mathbb{I}\bigotimes\mathbb{I} &0\\
%	-\mathbb{K}\bigotimes\mathbb{I}\bigotimes\mathbb{G} &0   & -\mathbb{L}\bigotimes\mathbb{I}\bigotimes\mathbb{I}
%	\end{pmatrix}	
%\end{eqnarray*}

%\section{Implementation}
%In this section, it will be discussed how the IP-DG scheme can be implemented for a domain $\Omega=[0,1]^3$. Let either

%-------------------------------------------
% Numerical Examples
%-------------------------------------------
\section{Numerical Examples}
\begin{itemize}
\item full grid accuracy
\item sparse grids accuracy
\item condition number
\item Adaptivity?
\end{itemize}
\subsection{Test 1}
Let $\Omega=[0,1]^3$ and the exact solution is given by
\begin{eqnarray}
\bu(x,y,z)=\begin{pmatrix}
\sin(\pi y)\sin(\pi z)\\
\sin(\pi z)\sin(\pi x)\\
\sin(\pi x)\sin(\pi y)
\end{pmatrix}.
\end{eqnarray}
We shall assume Dirichelet boundary condition for the test.



%%-------------------
%% Table
%%-------------------
%\begin{table}
%\caption{Numerical performance of discontinuous Galerkin methods on the full grids with $\omega=1.$}
%\begin{tabular}{cccccccc}\hline\hline
%$h$ & DOFs & $L_{\infty}$-error & $L^2$-error & $\lambda_{\max}$ & $\lambda_{\min}$ & Cond & Iter \\ \hline
%\multicolumn{8}{c}{$k=1$}\\ \hline
%$2^{-2}$   &1536   &5.6951e-03   &4.6338e-02 &57024.430978    &18.878869 &3020.542718   &54\\
%$2^{-3}$   &12288 &5.7293e-04   &1.1669e-02 &240478.526698   &18.944817 &12693.631670  &73\\
%$2^{-4}$   &98304 &5.2287e-05   &2.9251e-03 &977313.341427   &18.792018 &52006.833005  &131\\
%$2^{-5}$   &786432 &4.6595e-06   &7.3205e-04 &3926302.820517  &18.754528 &209352.263631 &273\\
%$2^{-6}$   &6291456 &4.1286e-07   &1.8310e-04 &15722813.057832 &18.743123 &838857.696443 &511\\ \hline
%\multicolumn{8}{c}{$k=2$}\\ \hline
%$2^{-2}$   &5184 &2.2612e-04   &2.7613e-03 &390653.841257   &18.762569  &20820.914159   &104 \\
%$2^{-3}$   &41472 &1.2008e-05   &3.5137e-04 &1631101.376555  &18.725176  &87107.398789   &138\\
%$2^{-4}$   &331776 &5.5528e-07   &4.4120e-05 &6605880.278280  &18.739728  &352506.726599  &290\\
%$2^{-5}$   &2654208 &2.4821e-08   &5.5220e-06 &26512507.597910 &18.739309  &1414807.078682 &571\\
%$2^{-6}$   &21233664 &1.1009e-09   &6.9069e-07 &106137293.680189 &18.739221 &5663911.628555 &772\\ \hline
%\multicolumn{8}{c}{$k=3$}\\ \hline
%$2^{-2}$ &12288       &1.0338e-05   &1.2191e-04 &1558862.024776     &18.729517 &83230.229223     &90\\
%$2^{-3}$ &98304       &2.6618e-07   &7.7191e-06 &6465795.265607     &18.739240 &345040.415361   &99\\
%$2^{-4}$ &786432     &6.1067e-09   &4.8612e-07 &25880500.920109   &18.739210 &1381088.168674 &20\\
%$2^{-5}$ &6291456   &1.3618e-10   &3.0310e-08 &104161989.481832 &18.739209 &5558505.186208 &18\\
%$2^{-6}$ &50331648 &3.0148e-12   &1.8927e-09 &208632188.971009 &18.739209 &11133457.722958 &9\\ \hline\hline
%\end{tabular}
%\end{table}
%
%%---------------------
%% Multiwavelet results
%%---------------------------
%\begin{table}
%\caption{Numerical performance of multiwavelet discontinuous Galerkin methods on the full grids with $\omega=1.$}
%\begin{tabular}{cccccccc}\hline\hline
%$h$ & DOFs & $L_{\infty}$-error & $L^2$-error & $\lambda_{\max}$ & $\lambda_{\min}$ & Cond & Iter \\ \hline
%\multicolumn{8}{c}{$k=1$}\\ \hline
%$2^{-2}$	&1536 &2.2689e-02   &4.6338e-02   &57000.606538        &18.878870   &3019.280619       &53\\
%$2^{-3}$   &12288 &5.5197e-03   &1.1669e-02   &240325.556323      &18.944817   &12685.557146    &73\\
%$2^{-4}$   &98304 &1.3721e-03   &2.9251e-03   &967594.006696      &18.792019   &51489.625308    &125\\
%$2^{-5}$   &786432 &3.4273e-04   &7.3205e-04   &3926302.538525    &18.754462   &209352.983582   &268\\
%$2^{-6}$ &6291456 5.4864e-08   6.9069e-07 106137172.243337 18.739221 5663905.148146 772\\   \hline
%\multicolumn{8}{c}{$k=2$}\\ \hline
%$2^{-2}$	&5184	&5.8236e-04   &2.7613e-03	&390473.280474 &18.762345 &20811.538802 &103\\
%$2^{-3}$	&41472	&7.1924e-05   &3.5137e-04    &1631109.965402 &18.725177 &87107.855082 &138\\
%$2^{-4}$	&331776	&6.8607e-06   &4.4120e-05    &6606018.620291 &18.739728 &352514.116290 &290\\
%$2^{-5}$	&2654208	&6.1784e-07   &5.5220e-06   &26512166.161984 &18.739309 &1414788.858359 &571\\
%$2^{-6}$	&21233664&6291456 8.5683e-05   1.8310e-04 15722803.088791 18.743123 838857.164450 499\\
%\hline
%\multicolumn{8}{c}{$k=3$}\\ \hline
%$2^{-2}$	&12288	&2.8315e-05   &1.2191e-04  &1558258.985809     &18.729508 &83198.075655     &90\\
%$2^{-3}$	&98304	&1.7596e-06   &7.7191e-06  &6463584.921308     &18.739240 &344922.462667   &98\\
%$2^{-4}$	&786432	&8.4061e-08   &4.8612e-07  &25880500.453577   &18.739210 &1381088.143797 &20\\
%$2^{-5}$	&6291456	&3.7866e-09   &3.0309e-08  &104162860.983888 &18.739209 &5558551.693146  &18\\
%$2^{-6}$	&50331648&1.6805e-10   &1.8926e-09  &208621237.804132 &18.739209 &11132873.324451 &9\\ \hline
%\end{tabular}
%\end{table}

%-------------------
% Table
%-------------------
\begin{table}
\caption{Numerical performance of discontinuous Galerkin methods on the full grids with $\omega=1.$}
\begin{tabular}{cccccccc}\hline\hline
$h$ & DOFs & $L_{\infty}$-error & $L^2$-error & $\lambda_{\max}$ & $\lambda_{\min}$ & Cond & Iter \\ \hline
\multicolumn{8}{c}{$k=1$}\\ \hline
$2^{-2}$   &1536       &5.70e-03   &4.63e-02 &57024.43       &18.88 &3020.54     &54\\
$2^{-3}$   &12288     &5.73e-04   &1.17e-02 &240478.53     &18.94 &12693.63   &73\\
$2^{-4}$   &98304     &5.23e-05   &2.93e-03 &977313.34     &18.79 &52006.83   &131\\
$2^{-5}$   &786432   &4.66e-06   &7.32e-04 &3926302.82   &18.75 &209352.26 &273\\
$2^{-6}$   &6291456 &4.13e-07   &1.83e-04 &15722813.06 &18.74 &838857.70 &511\\ \hline
\multicolumn{8}{c}{$k=2$}\\ \hline
$2^{-2}$   &5184         &2.26e-04   &2.76e-03 &390653.84       &18.76  &20820.91     &104 \\
$2^{-3}$   &41472       &1.20e-05   &3.51e-04 &1631101.38     &18.73  &87107.40     &138\\
$2^{-4}$   &331776     &5.55e-07   &4.41e-05 &6605880.28     &18.74  &352506.73   &290\\
$2^{-5}$   &2654208   &2.48e-08   &5.52e-06 &26512507.60   &18.74  &1414807.08 &571\\
$2^{-6}$   &21233664 &1.10e-09   &6.91e-07 &106137293.68 &18.74 &5663911.63  &772\\ \hline
\multicolumn{8}{c}{$k=3$}\\ \hline
$2^{-2}$ &12288       &1.03e-05   &1.22e-04 &1558862.02     &18.73 &83230.23      &90\\
$2^{-3}$ &98304       &2.66e-07   &7.72e-06 &6465795.27     &18.74 &345040.42    &99\\
$2^{-4}$ &786432     &6.11e-09   &4.86e-07 &25880500.92   &18.74 &1381088.17  &20\\
$2^{-5}$ &6291456   &1.36e-10   &3.03e-08 &104161989.48 &18.74 &5558505.19  &18\\
$2^{-6}$ &50331648 &3.01e-12   &1.89e-09 &208632188.97 &18.74 &11133457.72 &9\\ \hline\hline
\end{tabular}
\end{table}

%---------------------
% Multiwavelet results
%---------------------------
\begin{table}
\caption{Numerical performance of multiwavelet discontinuous Galerkin methods on the full grids with $\omega=1.$}
\begin{tabular}{cccccccc}\hline\hline
$h$ & DOFs & $L_{\infty}$-error & $L^2$-error & $\lambda_{\max}$ & $\lambda_{\min}$ & Cond & Iter \\ \hline
\multicolumn{8}{c}{$k=1$}\\ \hline
$2^{-2}$	&1536      &2.27e-02   &4.63e-02   &57000.61        &18.88   &3019.28       &53\\
$2^{-3}$   &12288    &5.52e-03   &1.17e-02   &240325.56      &18.94   &12685.56     &73\\
$2^{-4}$   &98304    &1.37e-03   &2.93e-03   &967594.01      &18.79   &51489.63     &125\\
$2^{-5}$   &786432  &3.43e-04   &7.32e-04   &3926302.54    &18.75   &209352.98   &268\\
$2^{-6}$   &6291456 &8.57e-05  &1.83e-04   &15722803.09  &18.74   &838857.16   &499\\   \hline
\multicolumn{8}{c}{$k=2$}\\ \hline
$2^{-2}$	&5184	&5.82e-04   &2.76e-03    &390473.28       &18.76 &20811.54     &103\\
$2^{-3}$	&41472	&7.19e-05   &3.51e-04    &1631109.97     &18.73 &87107.86     &138\\
$2^{-4}$	&331776	&6.86e-06   &4.41e-05    &6606018.62     &18.74 &352514.12   &290\\
$2^{-5}$	&2654208	&6.18e-07   &5.52e-06    &26512166.16   &18.74 &1414788.86 &571\\
$2^{-6}$	&21233664&5.49e-08  &6.91e-07   &106137172.24 &18.74 &5663905.15  &772\\
\hline
\multicolumn{8}{c}{$k=3$}\\ \hline
$2^{-2}$	&12288	  &2.83e-05   &1.22e-04  &1558258.99     &18.73 &83198.08     &90\\
$2^{-3}$	&98304	  &1.76e-06   &7.72e-06  &6463584.92     &18.74 &344922.46   &98\\
$2^{-4}$	&786432	  &8.41e-08   &4.86e-07  &25880500.45   &18.74 &1381088.14 &20\\
$2^{-5}$	&6291456	  &3.79e-09   &3.03e-08  &104162860.98 &18.74 &5558551.69  &18\\
$2^{-6}$	&50331648&1.68e-10   &1.89e-09  &208621237.80 &18.74 &11132873.32 &9\\ \hline
\end{tabular}
\end{table}


%---------------------
% Sparse Grids results
%---------------------------
\begin{table}
\caption{Numerical performance of multiwavelet discontinuous Galerkin methods on the sparse grids with $\omega=1.$}
\begin{tabular}{cccccccc}\hline\hline
$h$ & DOFs & $L_{\infty}$-error & $L^2$-error & $\lambda_{\max}$ & $\lambda_{\min}$ & Cond & Iter \\ \hline
\multicolumn{8}{c}{$k=1$}\\ \hline
2           &9.3940e-02   &1.2889e-01 &32687.766946 &4.276172 &7644.166171 &149 \\ \hline
\end{tabular}
\end{table}




\begin{thebibliography}{99}
\bibitem{Dolean2010}
V. Dolean, H. Fahs, L. Fezoui, and S. Lanteri. Locally implicit discontinous Galerkin method for time domain electromagnetics, J. Comput. Phys., 229 (2010): 512-526.

\bibitem{GuoCheng2016}
W. Guo and Y. Cheng. A Sparse Grid Discontinuous Galerkin Method for High-Dimensional Transport Equations and Its Application to Kinetic Simulations. SIAM J. Sci. Comput., 38(6), A3381-A3409. 

\bibitem{HoustonPerugiaSchneebeli2005b}
P. Houston, I. Perugia, A. Schneebeli, and D. Sch\"{o}tzau. Interior penalty method for the indefinite time-harmonic
Maxwell equations. Numer. Math., 100 (2005): 485-518.

\bibitem{Kopriva2002}
D.A.Kopriva, S.L.Woodruff and M.Y.Hussaini. Computation of electromagnetic scattering with a non-conforming discontinuous spectral element method, Int. J. Numr. Meth. Engng, 53 (2002): 105-122.

\bibitem{LiHesthaven2014}
J. Li and J.S.Hesthaven. Analysis and application of the nodal discontinuous Galerkin method for wave propagation in metamaterials, J. Comput. Phys., 258 (2014): 915-930.

\bibitem{ShuOsher1988}
C.W. Shu and S. Osher. Efficient implementation of essentially non-oscillatory shock-capturing schemes. J. Comput.
Phys., 77 (1988):439-471.


\bibitem{Warburton2000}
T. Warburton. Application of the discontinuous Galerkin method to Maxwell's equations using unstructured polymorphic hp-dinite elements, In Discontinuous Galerkin Methods, 2000, 451-458.

\bibitem{XieWangZhang2013}
Z. Xie, B. Wang and Z. Zhang. Space-Time discontinuous Galerkin method for Maxwell's equations, Commun. Comput.Phys., 14 (2013): 916-939.



\end{thebibliography}

\end{document}


