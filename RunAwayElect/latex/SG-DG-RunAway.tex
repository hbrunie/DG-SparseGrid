
\documentclass[final,leqno]{siamltex704}
%\documentclass[leqno]{siamltex704}
\usepackage{epsfig}
\usepackage{amsmath,bm}
\usepackage{amssymb}
\usepackage{float}
\usepackage{tikz}
\usepackage{graphicx}
\usepackage[notcite,notref]{showkeys}
\newtheorem{algorithm}{Weak Galerkin Algorithm}
\newcommand{\bq}{{\bf q}}
\newcommand{\bn}{{\bf n}}
\newcommand{\bx}{{\bf x}}
\newcommand{\bv}{{\bf v}}
\def\bbb{{\bf b}}
\def\T{{\mathcal T}}
\def\E{{\mathcal E}}
\def\V{{\mathcal V}}
\def\W{{\mathcal W}}
\def\l{{\langle}}
\def\r{{\rangle}}
\def\jump#1{{[\![#1[\!]}}
\def\bbF{{\bf F}}
\def\bbf{{\bf f}}
\def\bn{{\bf n}}
\def\bq{{\bf q}}
\def\bV{{\bf V}}
\def\bu{{\bf u}}
\def\bv{{\bf v}}
\def\bw{{\bf w}}
\def\br{{\bf r}}
\def\bs{{\bf s}}
\def\bbQ{\mathbb{Q}}
\def\bfQ{\bf{Q}}
\def\bcurl{ \textbf{curl }}
\def\bE{{\bf E}}
\def\bB{{\bf B}}
\def\bx{{\bf x}}
\def\bU{{\bf U}}
\def\bV{{\bf V}}
\def\bJ{{\bf J}}

\def\ljump{{[\![}}
\def\rjump{{]\!]}}

\def\lavg{{\{\!\{}}
\def\ravg{{\}\!\}}}

\def\aa{\mathfrak{a}}
\def\bbQ{\mathbb{Q}}
\newcommand{\pT}{{\partial T}}
\newtheorem{defi}{Definition}[section]
\def\3bar{{|\hspace{-.02in}|\hspace{-.02in}|}}
%\setlength{\textwidth}{6truein} \setlength{\textheight}{8truein}
%\voffset=-0.55truein
%\hoffset=-0.65truein
\renewcommand{\ldots}{\dotsc}
\setlength{\parskip}{1\parskip}

\title{Discontinuous Galerkin Sparse Grids Methods for Run-Away Electron Model}


\author{
}

\begin{document}

\maketitle

\begin{abstract}

\end{abstract}

\begin{keywords}

\end{keywords}

\begin{AMS}
Primary: 65N15, 65N30; Secondary: 35J50
\end{AMS}
\pagestyle{myheadings}

\section{Introduction}\label{Section:Introduction}
Let the domain $\Omega\times(0,T)$ with $\Omega=[-1,1]$, the following problem will be considered in this paper,
\begin{eqnarray}
&&\frac{\partial f}{\partial t} = \frac{\partial}{\partial x}(1-x^2)\frac{\partial f}{\partial x},\mbox{ in }\Omega\label{eq:pde}\\
&&(1-x^2)\frac{\partial f}{\partial x}|_{x=\pm 1} = 0.\label{eq:bc}
\end{eqnarray}

% LDG
\section{Local Discontinuous Galerkin Methods}
We rewrite Eq. (\ref{eq:pde}) as the following system
\begin{eqnarray}
f_t-q_x=0,\ q-(1-x^2)f_x=0.
\end{eqnarray}
First, we divide $[-1,1]$ into $N$ cells
\begin{eqnarray*}
-1=x_{1/2}\le x_{3/2}\le \cdots\le x_{N+1/2}=1,
\end{eqnarray*}
and denote
$$
I_j=(x_{j-1/2},x_{j+1/2}),\ x_j=(x_{j-1/2},x_{j+1/2})/2,\ h_j=x_{j+1/2}-x_{j-1/2}
$$
as the cells, cell centers and cell lengths.

Define the discontinuous Galerkin finite element space as
\begin{eqnarray}
V_h^k=\{v:v|_{I_j}\in P^k(I_j), j=1,\cdots,N\},
\end{eqnarray}
where $P^k(I_j)$ denotes the space of polynomials in $I_j$ of degree at most $k$.

The numerical scheme is to find $f_h,q_h\in V_h^k$ such that, for all test functions $w,p\in V_h^k$, we have
\begin{eqnarray}
&&\int_{I_j}\frac{\partial f_h}{\partial t}wdx+\int_{I_j}q_h\frac{\partial w}{\partial x}dx-\hat{Q}_{j+1/2}(w)^-_{j+1/2}+\hat{Q}_{j-1/2}(w)^+_{j-1/2}=0,\label{eq:ldg-1}\\
&&\int_{I_j}q_h p-2xf_h p+(1-x^2)f_h\frac{\partial p}{\partial x}dx-\hat{F}_{j+1/2}(p)^-_{j+1/2}+\hat{F}_{j-1/2}(p)^+_{j-1/2}=0.\label{eq:ldg-2}
\end{eqnarray}
Here we have to define the numerical fluxes $\hat{Q}_{j+1/2}$, $\hat{Q}_{j-1/2}$, $\hat{F}_{j+1/2}$, and $\hat{F}_{j-1/2}$. The first trial is the average flux (central flux)
\begin{eqnarray}
\hat{Q}_{j+1/2}&=&\frac{1}{2}\bigg((q_h)_{j+1/2}^++(q_h)_{j+1/2}^-\bigg),\\
 \hat{Q}_{j-1/2}&=&\frac{1}{2}\bigg((q_h)_{j-1/2}^++(q_h)_{j-1/2}^-\bigg)\\
\hat{F}_{j+1/2}&=&\frac{1-x_{j+1/2}^2}{2}\bigg((f_h)_{j+1/2}^++(f_h)_{j+1/2}^-\bigg),\\
 \hat{F}_{j-1/2}&=&\frac{1-x_{j-1/2}^2}{2}\bigg((f_h)_{j-1/2}^++(f_h)_{j-1/2}^-\bigg).
\end{eqnarray}
On the boundary, we shall define $\hat{Q}_{1/2}=0$, $\hat{Q}_{N+1/2}=0$, $\hat{F}_{1/2}=0$, $\hat{F}_{N+1/2}=0$.


Notice that, from Eq. (\ref{eq:ldg-2}), we can solve $q_h$ explicitly in terms of $f_h$, just by inverting the small matrix. It turns out that the LDG scheme (\ref{eq:ldg-1})-(\ref{eq:ldg-2}) with the central fluxes is stable and convergent, but it loses one order of accuracy, to only $\mathcal{O}(h^k)$ for the error measured in the $L^2$-norm, for odd $k$.

Thus, the matrix form reads
\begin{eqnarray*}
\begin{cases}
\dfrac{{\bf dF}_h}{dt} =& \mathbb{A}_{2}{\bf Q}_h \\
{\bf Q}_h &= \mathbb{A}_{1} {\bf F}_h.
\end{cases}
\end{eqnarray*}

%Replace $p=q_h$ and $w=f_h$, and thus (\ref{eq:ldg-1})-(\ref{eq:ldg-2}) imply
%\begin{eqnarray}
%\frac{1}{2}\frac{d}{dt}\int_{I_j}(f_h)^2dx+\frac{d}{dt}\int_{I_j}(q_h)^2dx
%\end{eqnarray}

%\section{Notations}
%%-----------------------------
%% Notations for DG
%%-----------------------------
%For piecewise functions, we further introduce the jumps and averages as follows. For any edge $e=K^+\cap K^-\in\mathcal{E}_h$, the jumps across $e$ are defined as
%\begin{eqnarray*}
%\ljump v\rjump=v^+-v^-,
%\end{eqnarray*}
%and the averages are
%\begin{eqnarray*}
%\lavg v\ravg=\frac{1}{2}\left(v^++v^-\right).
%\end{eqnarray*}




%% Error Estimate
%
%\section{Sparse Grids}
%In this section, we introduce the grids and the associated finite element space. First, denote the one dimensional hierarchical decomposition of piecewise polynomial space on the interval $[0,1]$. The grids are defined as the nested grids, where the $n$-th level grid $\Omega_n$ consists of $2^n$ uniform cells $I_j^n=(2^{-n}j,2^{-n}(j+1)],j=0,\dots,2^n-1$, for any $n\ge 0.$ Let 
%\begin{eqnarray*}
%V_k^n:=\{v:v\in P^k(I^n_j),\forall j=0,\dots,2^n-1\}
%\end{eqnarray*}
%be the usual piecewise polynomials of degree at most $k$ on the $n$-th level grid $\Omega_n$. Then, we have
%$$V^0_k\subset V^1_k\subset V^2_k\subset\cdots.$$
%We can now define the multiwavelet subspace $W^n_k,n=1,2,\cdots$ as the orthogonal complement of $V_k^{n-1}$ in $V_k^n$ with respect to the $L^2$ inner product on $[0,1]$, i.e.,
%$$V_k^{n-1}\bigoplus W_k^n=V_k^n,\ W_k^n\pm V_k^{n-1}.$$
%Here, we let $W_k^0:=V_k^0$, which is standard piecewise polynomial space of degree $k$ on $[0,1]$. The dimension of $W_k^n$ is $2^{n-1}(k+1)$ when $n\ge 1$, and $k+1$ when $n=0$. In summary, we have found a hierarchical representation of the standard piecewise polynomial space $V_k^n$ on $\Omega_n$ as $V_k^n=\bigoplus_{0\le j\le n}W_k^j$.
%
%For a multi-index $\bm{\alpha}=(\alpha_1,\alpha_2,\alpha_3)\in\mathbb{N}_0^3$, where $\mathbb{N}_0^3$ denotes the set of nonnegative integers, the $l^1$ and $l^\infty$ norms are defined as
%$$|\bm{\alpha}|_1:=\sum_{m=1}^3\alpha_m,\ |\bm{\alpha}|_{\infty}:=\max_{1\le m\le 3}\alpha_m.$$
%We denote ${\bf l}=(l_1,l_2,l_3)\in\mathbb{N}_0^3$ the mesh level in a multivariate sense. We define the tensor-product mesh grid $\Omega_{\bf l}=\Omega_{l_1}\bigotimes\Omega_{l_2}\bigotimes\Omega_{l_3}$ and the corresponding mesh size $h_{\bf l}=(h_{l_1},h_{l_2},h_{l_3})$. Based on the grid $\Omega_{\bf l},$ we denote by $I_{\bf j}^{\bf l}=\{{\bf x}:x_m\in(h_mj_m,h_m(j_m+1)),m=1,2,3\}$ and elementary cell, and
%$${\bf V}_k^{\bf l}:=\{\bv:\bv(\bx)\in [{P}^k(I_{\bf j}^{\bf l})]^3,0\le{\bf j}\le 2^{\bf l}-1\}=[V^{l_1}_{k,x_1}\times V^{l_2}_{k,x_2}\times V^{l_3}_{k,x_3}]^3$$
%the tensor-product piecewise polynomial space, where $P^k(I_{\bf j}^{\bf l})$ denotes the collection of polynomials of degree up to $k$ in each dimension on cell $I_{\bf j}^{\bf l}$. If we use equal mesh refinement of size $h_N=2^{-N}$ in each coordinate direction, the grid and space will be denoted by $\Omega_N$ and $\bV_k^N,$ respectively.
%
%Based on a tensor-product construction, the multi-dimensional increment space can be defined as
%$${\bf W}_k^{\bf l}=[W_{k,x_1}^{l_1}\times W_{k,x_2}^{l_2}\times W_{k,x_3}^{l_3}]^3.$$
%Therefore, space ${\bf V}_k^{\bf l}$ can be represented by
%$$\bV_k^{\bf l}=\bigoplus_{0\le j_1\le l_1,0\le j_2\le l_2,0\le j_3\le l_3}{\bf W}_k^{\bf j}.$$
%The standard tensor-product polynomial space $\bV_k^N$ and sparse finite element approximation space $\hat{\bV}_k^N$ on $\Omega_N$ are defined by
%\begin{eqnarray}
%\bV_k^N&=&\bigoplus_{|{\bf l}|_\infty\le N,{\bf l}\in\mathbb{N}_0^3}{\bf W}_k^{\bf l},\\
%\hat{\bV}_k^N&=&\bigoplus_{|{\bf l}|_1\le N,{\bf l}\in\mathbb{N}_0^3}{\bf W}_k^{\bf l}
%\end{eqnarray}
%The sparse finite element space $\hat{\bV}_k^N$ is a subset of $\bV_k^N$, and its number of degrees of freedom scales as $\mathcal{O}([3(k+1)^3(2N)N^{2}])$, which is significantly less than that of $\bV_k^N$ with exponential dependence on $[3(k+1)^32^{3N}]$. This is the key for computational savings in high dimensions.

\subsection{Temporal Discretizations}
We use total variation diminishing (TVD) high-order Runge-Kutta methods to solve the method of lines ODE resulting from the semi-discrete DG scheme, $\frac{d}{dt}G_h=R(G_h)$. Such time stepping methods are convex combinations of the Euler forward time discretization. The commonly used third-order TVD Runge-Kutta method is given by
\begin{eqnarray}
G_h^{(1)}&=&G_h^n+\Delta tR(G_h^n),\notag\\
G_h^{(2)}&=&\frac{3}{4}G_h^n+\frac{1}{4}G_h^{(1)}+\frac{1}{4}\Delta tR(G_h^{(1)}),\notag\\
G_h^{n+1}&=&\frac{1}{3}G_h^n+\frac{2}{3}G_h^{(2)}+\frac{2}{3}\Delta tR(G_h^{(2)}),
\end{eqnarray}
where $G_h^n$ represents a numerical approximation of the solution at discrete time $t_n$. A detailed description of the TVD Runge-Kutta method can be found in \cite{ShuOsher1988}.

\subsection{Numerical Experiments}
\subsubsection{Example 1} In this test, we choose a proper source term such that the exact solution is 
$$f(x,t)=\sin(\pi x)\exp(t).$$

\begin{figure}[h!]
\centering
\begin{tabular}{cc}
  \includegraphics[width=.45\textwidth]{./L2error}&
   \includegraphics[width=.45\textwidth]{./test1-Linftyerror} \\
  \footnotesize (a) & \footnotesize(b) 
\end{tabular}
\caption{Test 1: Convergence results of error measured in $L^2$-norm and $L_{\infty}$ norm.}
\end{figure}

\begin{figure}[h!]
\centering
\begin{tabular}{cc}
  \includegraphics[width=.45\textwidth]{./sol-f}&
   \includegraphics[width=.45\textwidth]{./sol-q} \\
  \footnotesize (a) & \footnotesize(b) 
\end{tabular}
\caption{Test 1: Plot for solutions (a) $f$; (b) $q$.}
\end{figure}

%CFL = 0.001 Deg = 2
%1.8296e-01   1.9889e-01 7.5308e-01   9.5192e-01
%7.7859e-02   9.5594e-02 3.8083e-01   4.6121e-01
%3.6593e-02   4.7238e-02 1.9399e-01   2.3431e-01
%1.7996e-02   2.3564e-02 9.7497e-02   1.1663e-01
%
%Deg = 3
%5.4127e-02   7.1077e-02 1.6754e-01   1.6957e-01
%2.7788e-03   3.6478e-03 1.3864e-02   1.4808e-02
%2.9264e-04   3.3966e-04 1.5324e-03   1.6930e-03
%3.5655e-05   3.7078e-05 1.8583e-04   2.0433e-04
%
%Deg = 4
%3.2418e-03   2.8363e-03 1.6971e-02   1.1397e-02
%4.6308e-04   4.3523e-04 1.7831e-03   1.5391e-03
%6.1735e-05   5.6997e-05 2.1482e-04   2.0341e-04
%7.8808e-06   7.2173e-06 2.6608e-05   2.5684e-05

\subsubsection{Example 2} In this test, we choose a proper source term such that the exact solution is 
$$f(x,t)=\cos(\pi x)\exp(t).$$

\begin{figure}[h!]
\centering
\begin{tabular}{cc}
  \includegraphics[width=.45\textwidth]{./test2-L2error}&
   \includegraphics[width=.45\textwidth]{./test2-Linftyerror} \\
  \footnotesize (a) & \footnotesize(b) 
\end{tabular}
\caption{Test 2: Convergence results of error measured in $L^2$-norm and $L_{\infty}$ norm.}
\end{figure}

\begin{figure}[h!]
\centering
\begin{tabular}{cc}
  \includegraphics[width=.45\textwidth]{./test2-sol-f}&
   \includegraphics[width=.45\textwidth]{./test2-sol-q} \\
  \footnotesize (a) & \footnotesize(b) 
\end{tabular}
\caption{Test 2: Plot for solutions (a) $f$; (b) $q$.}
\end{figure}

%Deg = 2
%   1.4887e-01   1.6477e-01 8.1850e-01   1.2631e+00
%   6.2397e-02   6.8786e-02 4.0606e-01   6.2627e-01
%   3.0414e-02   3.2788e-02 2.0076e-01   3.1026e-01
%   1.5335e-02   1.6164e-02 9.9869e-02   1.5443e-01
%Deg = 3   
%   3.0115e-02   3.9567e-02 1.4831e-01   1.9290e-01
%   2.5491e-03   3.2423e-03 1.4562e-02   1.3260e-02
%   2.9283e-04   3.0353e-04 1.6325e-03   1.7234e-03
%   3.5710e-05   3.4597e-05 1.9865e-04   1.9296e-04
%Deg = 4
%   2.1547e-03   1.6654e-03 1.9086e-02   2.0272e-02
%   2.1478e-04   1.7957e-04 2.3586e-03   2.8417e-03
%   2.5985e-05   2.3556e-05 2.8320e-04   3.5870e-04
%   3.2912e-06   2.9420e-06 3.4739e-05   4.4667e-05

\subsubsection{Example 3}
In this example, we shall consider the initial condition as follows:
\begin{eqnarray*}
f_0=\frac{A}{2\sinh A}\exp(Ax).
\end{eqnarray*}
Here we shall take $A=0.1,1,10$ for testing.
\begin{figure}[h!]
\centering
\begin{tabular}{c}
  \includegraphics[width=.9\textwidth]{./test3-Solution}\\
   \includegraphics[width=.9\textwidth,height=.6\textwidth]{./test3-TotalPartical} 
  %\footnotesize (a) & \footnotesize(b) 
\end{tabular}
\caption{Test 3: (a) Plot for solutions; (b) Total particles.}
\end{figure}

\subsubsection{Example 4}\label{Test:Diffusion-Test4}
In this test, we shall assume the initial condition as
\begin{eqnarray*}
f_0(x)=\sum_{L=0}^\infty h_LP_L(x),
\end{eqnarray*}
where $P_L$ is the Legendre polynomial of degree $L$ and the coefficients are $h_0=3,\ h_1=0.5,\ h_2=1,\ h_3=0.7,\ h_4=3,\ h_6=3,$ and all other $h_L$ are set as zero.
\begin{figure}[H]
\centering
\begin{tabular}{c}
  \includegraphics[width=.9\textwidth]{./Diff-Deg5_Lev4}
  %\footnotesize (a) & \footnotesize(b) 
\end{tabular}
\caption{Test \ref{Test:Diffusion-Test4}: Numerical solution with Lev = 4 and Deg = 5.}
\end{figure}


\section{Convection Equation}
In this section, we shall consider the following hyperbolic equations.
Let the domain $\Omega\times(0,T)$ with $\Omega=[-1,1]$, the following problem will be considered in this paper,
\begin{eqnarray}
&&\frac{\partial f}{\partial t} = -\frac{\partial}{\partial x}(1-x^2)f,\mbox{ in }\Omega\label{eq:hyper-pde}%\\
%&&(1-x^2)\frac{\partial f}{\partial x}|_{x=\pm 1} = 0.\label{eq:hyper-bc}
\end{eqnarray}
The DG method will be applied to approximate the above equation. The scheme will be described as follows:
\begin{eqnarray*}
\int_{I_j}\bigg(\frac{\partial f_h}{\partial t} w-(1-x^2)f_h\frac{\partial w}{\partial x}\bigg) dx + (\tilde{F}_{j+1/2})(w)_{j+1/2}^- -(\tilde{F}_{j-1/2})(w)_{j-1/2}^+ = 0.
\end{eqnarray*}
The upwind flux has been chosen as
$\tilde{F}_{j+1/2}=(1-x_{j+1/2})f_{j+1/2}^-$ and $\tilde{F}_{j-1/2}=(1-x_{j-1/2})f_{j-1/2}^-$.

Thus, the matrix form reads
\begin{eqnarray*}
\dfrac{{\bf dF}_h}{dt} =& \mathbb{B}{\bf F}_h.
\end{eqnarray*}

%---------------------------
% Numerical Example
%---------------------------
\subsection{Numerical Test}
Let 
\begin{eqnarray*}
f(x,t)&=&\frac{1-\phi^2}{1-x^2}f_0(\phi),\
\phi(x,t)=\tanh(\tanh^{-1}x-t).
\end{eqnarray*}

\subsubsection{Test 1}\label{Num-2}
In this test, we shall use $f_0=1$ as initial condition.

\begin{figure}[H]
\centering
\begin{tabular}{c}
  \includegraphics[width=.95\textwidth]{./hyper-cf-Test1_v2}
  \end{tabular}
\caption{Test \ref{Num-2}: Plot for solutions for Lev = 5, Deg = 5 of Central flux at time = 0.5, 1, 2, 3.}
\end{figure}


%\begin{figure}[H]
%\centering
%\begin{tabular}{c}
%  \includegraphics[width=.95\textwidth]{./hyper-cf-Test1}
%  \end{tabular}
%\caption{Test \ref{Num-2}: Plot for solutions for Lev = 5, Deg = 5 of Central flux at time = 0.5, 1, 2, 3.}
%\end{figure}

\begin{figure}[H]
\centering
\begin{tabular}{c}
  \includegraphics[width=.95\textwidth,height=.65\textwidth]{./hyper-uf-Test1_v2}
  \end{tabular}
\caption{Test \ref{Num-2}: Plot for solutions for Lev = 5, Deg = 5 of upwind flux at time = 0.5, 1, 2, 3.}
\end{figure}

%\begin{figure}[H]
%\centering
%\begin{tabular}{c}
%  \includegraphics[width=.95\textwidth]{./hyper-uf-Test1}
%  \end{tabular}
%\caption{Test \ref{Num-2}: Plot for solutions for Lev = 5, Deg = 5 of upwind flux at time = 0.5, 1, 2, 3.}
%\end{figure}


\subsubsection{Test 2}\label{Num-3}
In this test, we shall use $f_0=\exp(-x^2/\sigma^2),\ \sigma=0.1$ as initial condition. 
%k=2 T=0.5
%2 2.7454e-01   6.2949e-01

%\begin{verbatim}
%CF
%2 3.5893e-01   9.2580e-01
%3 1.5262e-01   4.4423e-01
%4 1.2866e-02   2.5458e-02
%5 2.8658e-04   8.9971e-04
%UF
%k=3, T=0.5
%2 1.8697e-01   5.2381e-01
%3 5.0580e-02   1.2419e-01
%4 6.5368e-03   1.6838e-02
%5 4.4662e-04   1.7248e-03
%\end{verbatim}

%\begin{figure}[H]
%\centering
%\begin{tabular}{c}
%  \includegraphics[width=.95\textwidth]{./hyperbolic-k2-compare}
%  \end{tabular}
%\caption{Test \ref{Num-3}: Plot for solutions for Lev = 4, T = 0.5.}
%\end{figure}
%
%\begin{figure}[H]
%\centering
%\begin{tabular}{c}
%  \includegraphics[width=.95\textwidth]{./hyperbolic-k4}\\
%  \includegraphics[width=.95\textwidth]{./hyperbolic-k4-zoom}
%  \end{tabular}
%\caption{Test \ref{Num-3}: Plot for solutions for Deg = 4, T = 0.5 for upwind flux.}
%\end{figure}
%
%\begin{figure}[H]
%\centering
%\begin{tabular}{c}
%  \includegraphics[width=.95\textwidth]{./hyperbolic-k4-CF}\\
%  \includegraphics[width=.95\textwidth]{./hyperbolic-k4-CF-Zoom}
%  \end{tabular}
%\caption{Test \ref{Num-3}: Plot for solutions for Deg = 4, T = 0.5 for central flux.}
%\end{figure}

\begin{figure}[H]
\centering
\begin{tabular}{c}
  \includegraphics[width=.95\textwidth,height=0.65\textwidth]{./hyper-CF-Test2_v2}
  \end{tabular}
\caption{Test \ref{Num-3}: Plot for solutions for Lev = 5, Deg = 5 of Central flux at time = 0.5, 1, 2, 3.}
\end{figure}


%\begin{figure}[H]
%\centering
%\begin{tabular}{c}
%  \includegraphics[width=.95\textwidth]{./hyper-cf-Test2}
%  \end{tabular}
%\caption{Test \ref{Num-3}: Plot for solutions for Lev = 5, Deg = 5 of Central flux at time = 0.5, 1, 2, 3.}
%\end{figure}

\begin{figure}[H]
\centering
\begin{tabular}{c}
  \includegraphics[width=.95\textwidth]{./hyper-UF-Test2_v2}
  \end{tabular}
\caption{Test \ref{Num-3}: Plot for solutions for Lev = 5, Deg = 5 of upwind flux at time = 0.5, 1, 2, 3.}
\end{figure}


Next, we test the adaptive method for this test. In this test, we have use the Hash table and leaf hash table. Hash table contains all the grids $\Omega^{n}_\ell$, and the leaf hash only consist of the grids without children. 

The adaptive algorithm is as follows: for chosen value of $\epsilon>0$ and $\eta>0$
\begin{itemize}
\item[Step 1.] We start with previous (initial) mesh and solution ${\bf F}^{n}$
\item[Step 2.] Pre-predict the solution by Euler time stepping method
$${\bf F}^{(p)} = {\bf F}^{n}+\Delta t \mathbb{B}{\bf F}^{n}$$
\item[Step 3.] Refinement: if the coefficient $|{\bf F}^{(p)}_j|>\epsilon$, we add additional resolution to the grid. The checking up procedure is for the global grids. But we only add the non-exist grid. Then the global grid and leaf grid will be modified accordingly.
\item[Step 4.] Use 3-rd Runge-Kutta method to compute ${\bf F}^{n+1}$
\item[Step 5.] Coarsening: if the coefficient of $|{\bf F}^{n+1}_j|>\eta$, we coarsen the grid. The checking up procedure is only carried out for the leaf grids. Then the global grid and leaf grid will be modified accordingly.
\item[Step 6.] Go to Step 2 until end of time stepping
\end{itemize}

\begin{figure}[H]
\centering
\begin{tabular}{c}
  \includegraphics[width=.95\textwidth,height=.6\textwidth]{./HyperAdap_Lev4_Deg2}
  \end{tabular}
\caption{Test \ref{Num-3}: Plot for solutions for Lev = 4, Deg = 2 for adaptive method.}
\end{figure}


%\begin{figure}[H]
%\centering
%\begin{tabular}{c}
%  \includegraphics[width=.95\textwidth]{./hyper-uf-Test2}
%  \end{tabular}
%\caption{Test \ref{Num-3}: Plot for solutions for Lev = 5, Deg = 5 of upwind flux at time = 0.5, 1, 2, 3.}
%\end{figure}

% Convection + Diffusion
\section{Convection and Diffusion equations}
Let
\begin{eqnarray*}
\frac{\partial f}{\partial t} =   C\frac{\partial}{\partial x}(1-x^2)\frac{\partial f}{\partial x}-E\frac{\partial}{\partial x}(1-x^2)f .
\end{eqnarray*}
First, we shall rewrite the equation as follows,
\begin{eqnarray*}
\frac{\partial f}{\partial t}+E\frac{\partial }{\partial x} (1-x^2)f &=& C\frac{\partial }{\partial x}q,\\
q - (1-x^2)\frac{\partial f}{\partial x}&=&0.
\end{eqnarray*}
Then the scheme becomes, to find $f_h,q_h\in V_h^k$, such that
\begin{eqnarray*}
&&\int_{I_j}\bigg(\frac{\partial f_h}{\partial t} - E (1-x^2) f_h \frac{\partial w}{\partial x}\bigg) dx +E\bigg( (\tilde{F}_{j+1/2})(w)^-_{j+1/2}-(\tilde{F}_{j-1/2})(w)^+_{j-1/2}\bigg)
\\
&&\quad\quad\quad\quad=-C\int_{I_j}q_h\frac{\partial w}{\partial x}dx + C\bigg( (\hat{Q}_{j+1/2})(w)^-_{j+1/2} -(\hat{Q}_{j-1/2})(w)^+_{j-1/2}\bigg),
\\
&&\int_{I_j}\bigg(q_h p-2xf_h p+(1-x^2)f_h\frac{\partial p}{\partial x}\bigg)dx-(\hat{F}_{j+1/2})(p)^-_{j+1/2}+(\hat{F}_{j-1/2})(p)^+_{j-1/2}=0,
\end{eqnarray*}
holds for $w,p\in V_h^k$, where $\tilde{F}_j$ is the upwind flux depending on the sign of $E$.
Here we shall assume $E\ge 0$ and take $\tilde{F}_{j+1/2}=(1-x_{j+1/2}^2)(f_h)_{j+1/2}^-$, $\tilde{F}_{j-1/2}=(1-x_{j-1/2}^2)(f_h)_{j-1/2}^-$, and alternating diffusion flux as follows,
\begin{eqnarray*}
\hat{Q}_{j+1/2}&=&(q_h)_{j+1/2}^+,\
 \hat{Q}_{j-1/2}=(q_h)_{j-1/2}^+
 \\
\hat{F}_{j+1/2}&=&{(1-x_{j+1/2}^2)}(f_h)_{j+1/2}^-,\
 \hat{F}_{j-1/2}={(1-x_{j-1/2}^2)}(f_h)_{j-1/2}^-.
\end{eqnarray*}

Thus, the matrix form reads
\begin{eqnarray*}
\begin{cases}
\dfrac{{\bf dF}_h}{dt} =& \mathbb{B}{\bf F}_h+\mathbb{A}_{2}{\bf Q}_h \\
{\bf Q}_h &= \mathbb{A}_{1} {\bf F}_h.
\end{cases}
\end{eqnarray*}
Thus, we can solve ${\bf F}^{n+1}$ explicitly by 3-rd Runge-Kutta methods for the following system:
\begin{eqnarray}
\frac{d{\bf F}_h}{dt} = R({\bf F}_h):= \mathbb{B}{\bf F}_h+\mathbb{A}_{2}\mathbb{A}_{1}{\bf F}_h.
\end{eqnarray}

%---------------------------
% Numerical Example
%---------------------------
\subsection{Numerical Test}
\subsubsection{Test 1}\label{Num-4}
In this test, we shall take domain as $\Omega=[-1,1]$, initial condition as $f_0(x)=1/2$  for testing.
\begin{figure}[H]
\centering
\begin{tabular}{c}
  \includegraphics[width=.95\textwidth]{./condiff-cf-E1}
  \end{tabular}
\caption{Test \ref{Num-4}: Plot for solutions for Lev = 4, Deg = 5 of Central flux at time = 0.5, 1, 2, 2.5, 3.}
\end{figure}

\begin{figure}[H]
\centering
\begin{tabular}{c}
  \includegraphics[width=.95\textwidth]{./condiff-uf-E1_v2}
  \end{tabular}
\caption{Test \ref{Num-4}: Plot for solutions for Lev = 4, Deg = 5 of upwind flux at time = 0.5, 1, 1.5, 2, 2.5, 3.}
\end{figure}

\begin{figure}[H]
\centering
\begin{tabular}{c}
  \includegraphics[width=.95\textwidth]{./condiff-uf-E2_v2}
  \end{tabular}
\caption{Test \ref{Num-4}: Plot for solutions for Lev = 4, Deg = 5 of upwind flux at time = 0.5, 1, 1.5, 2, 2.5, 3.}
\end{figure}

\begin{figure}[H]
\centering
\begin{tabular}{c}
  \includegraphics[width=.95\textwidth]{./condiff-uf-E4_v2}
  \end{tabular}
\caption{Test \ref{Num-4}: Plot for solutions for Lev = 4, Deg = 5 of upwind flux at time = 0.5, 1, 1.5, 2, 2.5, 3.}
\end{figure}

%\begin{figure}[H]
%\centering
%\begin{tabular}{c}
%  \includegraphics[width=.95\textwidth]{./condiff-uf-E1}
%  \end{tabular}
%\caption{Test \ref{Num-4}: Plot for solutions for Lev = 4, Deg = 5 of upwind flux at time = 0.5, 1, 1.5, 2, 2.5, 3.}
%\end{figure}
%
%\begin{figure}[H]
%\centering
%\begin{tabular}{c}
%  \includegraphics[width=.95\textwidth]{./condiff-uf-E2}
%  \end{tabular}
%\caption{Test \ref{Num-4}: Plot for solutions for Lev = 4, Deg = 5 of upwind flux at time = 0.5, 1, 1.5, 2, 2.5, 3.}
%\end{figure}
%
%\begin{figure}[H]
%\centering
%\begin{tabular}{c}
%  \includegraphics[width=.95\textwidth]{./condiff-uf-E4}
%  \end{tabular}
%\caption{Test \ref{Num-4}: Plot for solutions for Lev = 4, Deg = 5 of upwind flux at time = 0.5, 1, 1.5, 2, 2.5, 3.}
%\end{figure}

\section{Damping}

alpha = 0

4 2 1.1562e-02   1.2294e-02
5 2 2.9780e-03   3.0811e-03
6 2 7.5420e-04   7.6794e-04

alpha = 1
4 2 6.4948e-02   7.7383e-02
5 2 3.2825e-02   3.9181e-02
6 2 1.6423e-02   1.9561e-02

alpha = 0
4 3 3.3562e-04   3.7930e-04
5 3 4.1533e-05   4.8142e-05
6 3 5.1423e-06   6.1385e-06





\begin{thebibliography}{99}

\bibitem{ShuOsher1988}
C.W. Shu and S. Osher. Efficient implementation of essentially non-oscillatory shock-capturing schemes. J. Comput.
Phys., 77 (1988):439-471.

\end{thebibliography}

\end{document}

%\begin{itemize}

\section{Numerical Experiments}
In this section, we perform several numerical experiments to validate the proposed numerical scheme. 

\subsection{Test 1}
Let $\Omega=[0,1]^3$, and the exact solutions are described as follows:
\begin{eqnarray}
\bE(x,y,z,t)=\begin{pmatrix}
-\cos(2\pi x)\sin(2\pi y)\sin(2\pi z)\cos(\omega t)\\
0\\
\sin(2\pi x)\sin(2\pi y)\cos(2\pi z)\cos(\omega t)
\end{pmatrix},\\
\bB(x,y,z,t)=\begin{pmatrix}
{-2\pi\sin(2\pi x)\cos(2\pi y)\cos(2\pi z)\sin(\omega t)}/{\omega}\\
{\quad 4\pi\cos(2\pi x)\sin(2\pi y)\cos(2\pi z)\sin(\omega t)}/{\omega}\\
{-2\pi\cos(2\pi x)\cos(2\pi y)\sin(2\pi z)\sin(\omega t)}/{\omega}
\end{pmatrix}
\end{eqnarray}

\subsection{Test 2}
For a domain $[0,1]^2$, the analytic solutions are taken as
\begin{eqnarray}
\bE(x,y,t)&=&\begin{pmatrix}
0\\
0\\
\cos(2\pi x)\cos(2\pi y)\cos(2\pi t)
\end{pmatrix},\\
\bB(x,y,t)&=&\begin{pmatrix}
\cos(2\pi x)\sin(2\pi y)\sin(2\pi t)\\
-\sin(2\pi x)\cos(2\pi y)\sin(2\pi t)\\
0
\end{pmatrix}.
\end{eqnarray}


2.0000e+00   2.0000e+00   5.2374e-03   9.0116e-03
2.0000e+00   3.0000e+00   5.4354e-03   1.2929e-02
2.0000e+00   4.0000e+00   2.5848e-03   9.0245e-03
2.0000e+00   5.0000e+00   1.3511e-03   5.0283e-03
2.0000e+00   6.0000e+00   5.2297e-04   2.6385e-03
2.0000e+00   7.0000e+00   1.8907e-04   1.3453e-03
2.0000e+00   8.0000e+00   6.7222e-05   6.7739e-04


% Table generated by Excel2LaTeX from sheet 'Sheet1'
\begin{table}[htbp]
  \centering
  \caption{Add caption}
    \begin{tabular}{rrrrr}
1/h & Max Error & Rate & L2 Error & Rate\\
    4     & 5.2374E-03 &       & 9.0116E-03 &  \\
    8     & 5.4354E-03 & - & 1.2929E-02 & - \\
    16   & 2.5848E-03 & 1.07  & 9.0245E-03 & 0.52 \\
    32   & 1.3511E-03 & 0.94  & 5.0283E-03 & 0.84 \\
    64   & 5.2297E-04 & 1.37  & 2.6385E-03 & 0.93 \\
    128 & 1.8907E-04 & 1.47  & 1.3453E-03 & 0.97 \\
    256 & 6.7222E-05 & 1.49  & 6.7739E-04 & 0.99 \\
    \end{tabular}%
  \label{tab:addlabel}%
\end{table}%


\begin{table}[htbp]
  \centering
  \caption{High-order Maxwell2}
    \begin{tabular}{rrrrr}
1/h & Max Error & Rate & L2 Error & Rate\\
\multicolumn{5} {c}{k=1}	\\
4	&5.2374E-03	&		&9.0116E-03	&\\
8	&5.4354E-03	&-0.05	&1.2929E-02	&-0.52\\
16	&2.5848E-03	&1.07	&9.0245E-03	&0.52\\
32	&1.3511E-03	&0.94	&5.0283E-03	&0.84\\
64	&5.2297E-04	&1.37	&2.6385E-03	&0.93\\
128	&1.8907E-04	&1.47	&1.3453E-03	&0.97\\
256	&6.7222E-05	&1.49	&6.7739E-04	&0.99\\
\multicolumn{5} {c}{k=2}	\\				
4	&3.3860E-03	&�		&6.0936E-03	&\\
8	&8.4866E-04	&2.00	&2.6162E-03	&1.22\\
16	&7.3711E-05	&3.53	&3.8082E-04	&2.78\\
32	&9.0947E-06	&3.02	&5.4754E-05	&2.80\\
64	&1.2368E-06	&2.88	&6.8743E-06	&2.99\\
128	&1.5210E-07	&3.02	&8.5553E-07	&3.01\\
\multicolumn{5} {c}{k=3}	\\					
4	&1.3336E-04	&�		&4.9267E-04	&\\
8	&1.6770E-05	&2.99	&6.2584E-05	&2.98\\
16	&2.1838E-06	&2.94	&7.7980E-06	&3.00\\
32	&2.6047E-07	&3.07	&9.6225E-07	&3.02\\
64	&3.2659E-08	&3.00	&1.2228E-07	&2.98\\
    \end{tabular}%
\end{table}%





